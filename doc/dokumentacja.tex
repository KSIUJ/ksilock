%%%%%%%%%%%%%%%%%%%%%%%%%%%%%%%%%%%%%%%%%
% Short Sectioned Assignment
% LaTeX Template
% Version 1.0 (5/5/12)
%
% This template has been downloaded from:
% http://www.LaTeXTemplates.com
%
% Original author:
% Frits Wenneker (http://www.howtotex.com)
%
% License:
% CC BY-NC-SA 3.0 (http://creativecommons.org/licenses/by-nc-sa/3.0/)
%
%%%%%%%%%%%%%%%%%%%%%%%%%%%%%%%%%%%%%%%%%

%----------------------------------------------------------------------------------------
%	PACKAGES AND OTHER DOCUMENT CONFIGURATIONS
%----------------------------------------------------------------------------------------

\documentclass[paper=a4, fontsize=11pt]{scrartcl} % A4 paper and 11pt font size

\usepackage[T1]{fontenc} % Use 8-bit encoding that has 256 glyphs
% \usepackage{fourier} % Use the Adobe Utopia font for the document - comment this line to return to the LaTeX default
\usepackage[utf8]{inputenc}
\usepackage{polski}
\usepackage[polish]{babel}

\usepackage{amsmath,amsfonts,amsthm} % Math packages

\usepackage{lipsum} % Used for inserting dummy 'Lorem ipsum' text into the template

\usepackage{sectsty} % Allows customizing section commands
\allsectionsfont{\centering \normalfont\scshape} % Make all sections centered, the default font and small caps

\usepackage{fancyhdr} % Custom headers and footers
\pagestyle{fancyplain} % Makes all pages in the document conform to the custom headers and footers
\fancyhead{} % No page header - if you want one, create it in the same way as the footers below
\fancyfoot[L]{} % Empty left footer
\fancyfoot[C]{} % Empty center footer
\fancyfoot[R]{\thepage} % Page numbering for right footer
\renewcommand{\headrulewidth}{0pt} % Remove header underlines
\renewcommand{\footrulewidth}{0pt} % Remove footer underlines
\setlength{\headheight}{13.6pt} % Customize the height of the header

\numberwithin{equation}{section} % Number equations within sections (i.e. 1.1, 1.2, 2.1, 2.2 instead of 1, 2, 3, 4)
\numberwithin{figure}{section} % Number figures within sections (i.e. 1.1, 1.2, 2.1, 2.2 instead of 1, 2, 3, 4)
\numberwithin{table}{section} % Number tables within sections (i.e. 1.1, 1.2, 2.1, 2.2 instead of 1, 2, 3, 4)

\setlength\parindent{0pt} % Removes all indentation from paragraphs - comment this line for an assignment with lots of text

%----------------------------------------------------------------------------------------
%	TITLE SECTION
%----------------------------------------------------------------------------------------

\newcommand{\horrule}[1]{\rule{\linewidth}{#1}} % Create horizontal rule command with 1 argument of height

\title{	
\normalfont \normalsize 
\textsc{Koło Studentów Informatyki Uniwersytetu Jagiellońskiego} \\ [25pt] % Your university, school and/or department name(s)
\horrule{0.5pt} \\[0.4cm] % Thin top horizontal rule
\huge Dokumentacja techniczna zamka elektronicznego \\ % The assignment title
\horrule{2pt} \\[0.5cm] % Thick bottom horizontal rule
}

\author{} % Your name

\date{\normalsize5 maja 2015} % Today's date or a custom date

\begin{document}

\maketitle

Niniejsza dokumentacja stanowi załącznik do podania, datowanego na 5 maja 2015 roku, do Pana Dziekana Macieja Ślusarka.

\section{Wstęp}

Na wstępie chcemy umotywować chęć zainstalowania zamka elektronicznego własnej konstrukcji. Posiadanie takiego zamka w siedzibie koła
umożliwi nam łatwiejszą kontrolę nad tym, kto przebywa w kole oraz uprości zarządzanie kluczem - nie będzie więcej zachodzić potrzeba
wypożyczania klucza z portiernii wydziału pod zastaw legitymacji, co obecnie sprawia drobne kłopoty. Proponujemy własną
konstrukcję zamka, ponieważ jako kart dostępowych chcemy użyć Elektronicznej Legitymacji Studenckiej (zwanej dalej ELS), którą ma każdy
członek naszego koła - dzięki temu zminimalizujemy koszty wdrożenia. Niestety, wyklucza to zastosowanie systemu, który obecnie jest używany
na wydziale, ponieważ z tego co wiemy, używa on kart innego typu, które nie są kompatybilne z ELS. Oprócz tego, możemy stworzyć system,
który będzie w pełni spełniał nasze oczekiwania, przykładowo będzie łączył się z centralnie zarządzaną bazą użytkowników osadzoną w naszej
serwerowni.

%------------------------------------------------

\section{Zasada działania}

Zamek będzie pracował w trybie standardowo zamkniętym, tzn. w wypadku braku prądu drzwi pozostaną zamknięte.

\paragraph{Otwieranie z zewnątrz}

Istnieją dwie możliwości - za pomocą karty lub za pomocą klucza. W przypadku karty wystarczy przyłożyć ją do czytnika, ten po dokonaniu
autoryzacji zwolni blokadę drzwi podając napięcie na elektrozaczep. Kluczem dalej będzie można otworzyć drzwi przekręcając go w zamku do
oporu.

\paragraph{Otwieranie z wewnątrz}

Wystarczy nacisnąć klamkę, ponieważ elektrozaczep blokuje język zamka, który się chowa. Ten fakt sprawia, że proponowane przez nas rozwiązanie
jest zupełnie bezpieczne i nie powoduje żadnych dodatkowych kłopotów w przypadku potrzeby szybkiego opuszczenia pomieszczenia.

\paragraph{Zamykanie}

Drzwi wystarczy zatrzasnąć. Niepożądane jest zamykanie na klucz, co jest oczywiście dalej możliwe, jednak powoduje to stałą blokadę drzwi.

%------------------------------------------------

\section{Moduły}

\paragraph{Czytnik}

Czytnik kart będzie oparty o układ PN532, który potrafi odczytywać karty pracujące na częstotliwości 13,56MHz (czyli ELS), przetestowaliśmy moduł
ITEAD PN532 NFC.

\paragraph{Mikrokontroler}

Jako mikrokontrolera zarządzającego zamkniem chcemy użyć jednej z mniejszych wersji Arduino, np. Arduino Nano. Zamierzamy dobrać do
niego moduł ethernetu, aby umożliwić komunikację z zewnętrznym serwerem. Cała komunikacja będzie się odbywać za pomocą SPI.

\paragraph{Elektrozaczep}

Wybór konkretnego modelu elektrozaczepu zamierzamy omówić z Administratorem WMiI.

\section{Potrzebne modyfikacje}

Wprowadzenie systemu wymaga kilku modyfikacji istniejącej infrastruktury:
\begin{itemize}
 \item zainstalowania elektrozaczepu we framudze drzwi
 \item wyprowadzenia na zewnątrz czytnika kart
 \item doprowadzenia zasilania
 \item zlikwidowania zewnętrznej klamki
 \item doprowadzenia ethernetu (możliwe jest położenie go z wnętrza koła)
\end{itemize}

\end{document}